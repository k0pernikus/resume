% resume.tex
% vim:set ft=tex spell:

\documentclass[11pt,a4paper]{article}
\usepackage[a4paper,margin=2cm]{geometry}
\usepackage[utf8]{inputenc}
\usepackage{mdwlist}
\usepackage[T1]{fontenc}
\usepackage{textcomp}
\usepackage{tgpagella}
\usepackage[hidelinks]{hyperref}

\renewcommand*\rmdefault{iwona}
\pagestyle{empty}
\setlength{\tabcolsep}{0em}

% indentsection style, used for sections that aren't already in lists
% that need indentation to the level of all text in the document
\newenvironment{indentsection}[1]%
{\begin{list}{}%
	{\setlength{\leftmargin}{#1}}%
	\item[]%
}
{\end{list}}

% opposite of above; bump a section back toward the left margin
\newenvironment{unindentsection}[1]%
{\begin{list}{}%
	{\setlength{\leftmargin}{-0.5#1}}%
	\item[]%
}
{\end{list}}

% format two pieces of text, one left aligned and one right aligned
\newcommand{\headerrow}[2]
{\begin{tabular*}{\linewidth}{l@{\extracolsep{\fill}}r}
	#1 &
	#2 \\
\end{tabular*}}

% make "C++" look pretty when used in text by touching up the plus signs
\newcommand{\CPP}
{C\nolinebreak[4]\hspace{-.05em}\raisebox{.22ex}{\footnotesize\bf ++}}

% and the actual content starts here
\begin{document}

\begin{center}
	{\LARGE \textbf{Philipp Kretzschmar}}
	  
	Herrmann-Kauffmann-Str. 16b\ \ \textbullet
	\ \ 22307 Hamburg
	\\
	+491577-3874337\ \ \textbullet
	\ \ philipp.kretzschmar@gmail.com
\end{center}

\hrule
\vspace{-0.4em}
\subsection*{Tätigkeit}

\begin{itemize}
	\parskip=0.1em
	
	\item
	\headerrow
	{\textbf{pilot Co & KG GmbH, ehemals Spot Media AG}}
	{\textbf{Hamburg}}
	\\
	\headerrow
	{\emph{Junior Web Developper}}
	{\emph{2013-07 -- 2013-09}}
	\begin{itemize*}
		\item Erstellung von Facebook Gewinnspiel Apps
		\item Wartung eines eigenentickelten PHP-Frameworks
    \item Templating nach strikten Kundenvorgaben unter Berücksichtigung verschiedener Browser
		\item JavaScript-App
		\item Freistellen von Bildern mittels PhotoShop
	\end{itemize*}
\end{itemize}

\begin{itemize}
	\parskip=0.1em
	
	\item
	\headerrow
	{\textbf{Commerce Plus+, ehemals Spot Media AG}}
	{\textbf{Hamburg}}
	\\
	\headerrow
	{\emph{Auszubildender zum Anwendungsentwickler Fachinformatiker}}
	{\emph{2010 -- 2013-01-09}}
	\begin{itemize*}
		\item Erstellung und Wartung von Symonfy2, Symfony1.4-Apps
		\item Wartung bestehender Shop-Lösungen
		\item Erstellen von Gewinnspielaktionen
		\item Templating nach strikten Kundenvorgaben unter Berücksichtigung verschiedener Browser
		\item Mitarbeit an Facebook-Apps
		\item JavaScript-Apps, bspw. Audio- u. Video-Player und Einbindung von Google Maps API
		\item Datenimport aus XML- und JSON-Quellen in Datenbanken via PHP
		\item Dokumentationserstellung
	\end{itemize*}
\end{itemize}

\hrule
\vspace{-0.4em}
\subsection*{Bildung}

\begin{itemize}
	\parskip=0.1em
	
	\item 
	\headerrow
	{\textbf{Staatliche Gewerbeschule G18}}
	{\textbf{Hamburg}}
	\\
	\headerrow
	{\emph{Schuliche Ausbildung zum Fachinformatiker Anwendungsentwicklung}}
	{\emph{2010 -- 2013}}
	\begin{itemize*}
		\item HWK-Abschlussprüfung 83/100 Punkte (gut)
		\item Gesamtdurchschnitt: 1,5
		\item Besonderes Projekt: \\ In Gruppenarbeit Erstellung einer Android-App mit selbstgewähltem Thema.
	\end{itemize*}
		
		  
	\item 
	\headerrow
	{\textbf{Universität Leipzig}}
	{\textbf{Leipzig}}
	\\
	\headerrow
	{\emph{Philosophie B.A., abgebrochen}}
	{\emph{2007 -- 2010}}
	
	\item 
	\headerrow
	{\textbf{Universität Leipzig}}
	{\textbf{Leipzig}}
	\\
	\headerrow
	{\emph{Informatik B.S., abgebrochen}}
	{\emph{2006 -- 2007}}
	
  \item 
	\headerrow
	{\textbf{Gymnasium Delitzsch}}
	{\textbf{Delitzsch}}
	\\
	\headerrow
	{\emph{Abitur}}
	{\emph{1998 -2006}}
	
\end{itemize}

\hrule
\vspace{-0.4em}
\subsection*{Technische Fähigkeiten}

\begin{indentsection}{\parindent}
	\hyphenpenalty=1000
	\begin{description*}
		\item[Programmiersprachen:]
		PHP5.3/5.4, JavaScript, Python, Java, shell script, SQL
		\item[Frameworks:]
		symfony2, symfony1.4, jQuery, bottle, Django, Magento, Wordpress, Android SDK
		\item[Templating-Sprachen:]
		html5, sass, less, css3, twig, Jinja2, markup, \LaTeX
		\item[Konzepte:]
		MVC, OOP, ORM, Design Patterns, Clean Code
		\item[Werkzeuge:]
		git, svn, phpstorm, pycharm, intelliJ, bash, propel, doctrine2, sqlite, mysql
	\end{description*}
\end{indentsection}


\hrule
\vspace{-0.4em}
\subsection*{Fremdsprachen}

\begin{indentsection}{\parindent}
	\hyphenpenalty=1000
	\begin{description*}
		\item[Englisch:] ~\\ sehr gut in Wort und Schrift (18 Jahre Erfahrung) \\ KMK-Fremdsprachenzertifikat Stufe C1 (höchstmöglichstes Niveau an der G18) \\ Englisch Leistungskurs in der Sekundarstufe II
		\item[Französisch:] ~\\ Grundkenntnisse
	\end{description*}
\end{indentsection}

\hrule
\vspace{-0.4em}
\subsection*{Sonstiges}

\begin{indentsection}{\parindent}
	\hyphenpenalty=1000
	\begin{description*}
	  \item[Github:] \url{https://github.com/k0pernikus/}
		\item[Wehrdienst:] ausgemustert
	\end{description*}
\end{indentsection}

\end{document}
